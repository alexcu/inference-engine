\section{Testing}
\label{sec:Testing}

\subsection{Unit Testing}
\label{sub:Unit Testing}

The codebase was developed using a test-driven develop strategy. Test coverage
for the majority of the codebase was factored in. All tests are located within
the the \texttt{test} subdirectory.

\subsection{Extended Parsing Entailment}
\label{sub:Extended Parsing Entailment}

\subsubsection{Truth Table Entailment}
\label{subs:Truth Table Entailment}

Extended propositional logic parsing was initially tested using the truth
table parser. An that was used in the testing strategy was the \emph{Smoke, Heat and Fire}
example that was provided in a previous tutorial. For example:
\begin{align*}
  KB     =& \; \{ ((Smoke \wedge Heat) \Rightarrow Fire) \Leftrightarrow ((Smoke \Rightarrow Fire) \vee (Heat \Rightarrow Fire)) \}\\
  \alpha =& \; Smoke
\end{align*}
This would produce a truth table similar to that shown in Table~\ref{tab:smoketable}.
As $r_{5}$ shows that the all models in the knowledge base holds, there would be
only four models in this entailment (the underlined $true$s in the $Smoke$ column).
Hence, it is shown that $KB \models \alpha$, and therefore output of the program is:

\begin{verbatim}
    YES: 4
\end{verbatim}

This is tested under the unit tests which are described in the
\texttt{TruthTableTests.swift} and \texttt{XCTestCase+EntailmentTest.swift}
files under the \texttt{test} directory.

While this test only contains one sentence in the knowledge base, it is to be
noted that this test aims to only test the extended parsing entailment using the
truth table method, and not the truth table method itself. There already exists
test which are to test the logic for the truth table, which are described in the
aformentioned files.

\begin{sidewaystable}
    \centering
    \caption{Truth table describe the \emph{Smoke, Heat and Fire} example}
    \label{tab:smoketable}
    \begin{tabular}{l|l|l||l|l|l|l|l|l}
    \hline
    \multirow{2}{*}{\textbf{$Smoke$}} & \multirow{2}{*}{\textbf{$Heat$}} & \multirow{2}{*}{\textbf{$Fire$}} & \textbf{$r_{0}$}             & \textbf{$r_{1}$}                  & \textbf{$r_{2}$}                  & \textbf{$r_{3}$}                 & \textbf{$r_{4}$}            & \textbf{$r_{5}$}                       \\ \cline{4-9}
                                      &                                  &                                  & \textbf{$Smoke \wedge Heat$} & \textbf{$r_{0} \Rightarrow Fire$} & \textbf{$Smoke \Rightarrow Fire$} & \textbf{$Heat \Rightarrow Fire$} & \textbf{$r_{2} \vee r_{4}$} & \textbf{$r_{1} \Leftrightarrow r_{4}$} \\ \hline
    $false$                           & $false$                          & $false$                          & $false$                      & $true$                            & $true$                            & $true$                           & $true$                      & $true$                                 \\
    $false$                           & $false$                          & $true$                           & $false$                      & $true$                            & $true$                            & $true$                           & $true$                      & $true$                                 \\
    $false$                           & $true$                           & $false$                          & $false$                      & $true$                            & $true$                            & $false$                          & $true$                      & $true$                                 \\
    $false$                           & $true$                           & $true$                           & $false$                      & $true$                            & $true$                            & $true$                           & $true$                      & $true$                                 \\
    \underline{$true$}                & $false$                          & $false$                          & $false$                      & $true$                            & $false$                           & $true$                           & $true$                      & $true$                                 \\
    \underline{$true$}                & $false$                          & $true$                           & $false$                      & $true$                            & $true$                            & $true$                           & $true$                      & $true$                                 \\
    \underline{$true$}                & $true$                           & $false$                          & $true$                       & $false$                           & $false$                           & $false$                          & $false$                     & $true$                                 \\
    \underline{$true$}                & $true$                           & $true$                           & $true$                       & $true$                            & $true$                            & $true$                           & $true$                      & $true$                                 \\ \hline
    \end{tabular}
\end{sidewaystable}

\subsubsection{Resolution Tests}
\label{subs:Resolution Tests}

Testing the resolution method identified several bugs with the NNF and CNF implementations.
Initially, NNF was not removing implications and biconditionals. It was found that
the NNF, CNF and \texttt{withoutImplications} methods were not recursively recalling
themselves if \emph{no} special conditions were true. Thus, rather than just
returning the result, the result is recreated using the same connecitve but applying
NNF or CNF to the lefthand and righthand sentences, when applicable. This key missing
implementation would cause errors in creating CNF, especially for the addition of
new cases.

The resolution method also uses tests to verify that the resolve method works as
intended. As seen in \texttt{ResolutionTests.swift}, there are tests for checking
resolve works for tautologies (i.e., $A \vee \neg A$) to the False propositional.
This was also identified to not work initially in the tests, as it was found that
a \texttt{Sentence}'s \texttt{join} method did not support the creation of the
false or true atom when appopriate. Since ``the empty clause—a disjunction of no disjuncts—is equivalent to False because a disjunction is true only if at least one of its disjuncts is true'' \citep[p.254]{aima2009}
it is required to be calculated correctly, in order to verify that resolution
method is complete, as shown in \texttt{Resolution.swift}. This is implemented
in the \texttt{\_ArrayType} extension applied to \texttt{Sentence}s, which can
be seen at the bottom of the \texttt{Sentence.swift} source.
