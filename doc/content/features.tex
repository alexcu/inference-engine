\section{Features}

A more extensive list of search features can be shown using the \verb`--help`
switch.

\subsection{Search Algorithms}
\label{sub:Search Algorithms}

The solver implements the following entailment methods:

\begin{itemize}
  \item Truth Table Method, \texttt{TT},
  \item Forward Chaining Method, \texttt{FC},
  \item Backward Chaining Method, \texttt{BC}, and
  \item Resolution Method, \texttt{RE}
\end{itemize}

\subsection{Extended Propositional Logic Token Parser}
\label{subs:tokenparser}

Full token parsing is used when parsing the knowledge base from the source file.
Thus, in addition to the horn-clause connectives for implicate (\texttt{=>}) and
for conjunction (\texttt{\&}), a a \textbf{biconditional} can be used using the
\texttt{<=>} syntax, a \textbf{disjunction} can be used using \texttt{\textbackslash/} or
a pipe \texttt{|}, and \textbf{negation} can be used using a tilde, \texttt{\textasciitilde}. In addition,
the tokeniser supports parentheses to associate precedence. Refer to the example
shown in Figure~\ref{fig:parsing}.

\begin{figure*}[h]
  \begin{subfigure}[b]{\textwidth}
    \centering
    $P \vee Q \wedge (\neg S \Rightarrow ((T \vee W) \wedge P)) \Leftrightarrow Q$
    \caption{Input logic}
    \label{fig:parsing-logic}
  \end{subfigure}\\\\
  \begin{subfigure}[b]{\textwidth}
    \centering
    \texttt{
      (P | (Q \& (\textasciitilde S => ((T | W) \& P)))) <=> Q
    }
    \caption{Parsed representation of input}
    \label{fig:parsing-parsed}
  \end{subfigure}
  \caption{Representation of full token parsing}
  \label{fig:parsing}
\end{figure*}

As shown in Figure~\ref{fig:parsing-parsed}, the correct operator precdence has
been used---the parser associates $Q \wedge (\neg S \Rightarrow ((T \vee W) \wedge P))$
over $P \vee Q$ shown by the output parentheses. However, $T \vee W$ takes
precedence over $W \wedge P$ as it is in a set of parentheses unless overriden by braces.

\subsection{Conjunctive and Negation Normal Form}
\label{subs:CNF}

To properly implement the Resolution Method, sentences must be resolved to their
conjunctive normal form and, thus, their negation normal form. As such, two
computed properties exist on all sentences:

\begin{itemize}
  \item \texttt{inConjunctiveNormalForm} (Listing~\ref{lst:tocnf}), and
  \item \texttt{inNegationNormalForm} (Listing~\ref{lst:tonnf})
\end{itemize}

When converting to conjunctive normal form, it is also required that all biconditionals
and implications are eliminated. This is implemented using the \texttt{withoutImplications}
computed property (Listing~\ref{lst:withoutimplications}). Note that the listings shown below
are only for \emph{complex} sentences; \textit{atomic} sentences have a default
implementation that just return the sentence unchanged. Custom operators defined
in the code make it easier to interoperate sentences together; refer to the implication
and biconditional eliminations in Listing~\ref{lst:withoutimplications}.

\lstinputlisting[
  label=lst:withoutimplications,
  caption=
    Eliminating all implications and biconditionals in a complex sentence,
  language=swift,
  linerange=108-129
]{../../src/ComplexSentence.swift}

\lstinputlisting[
  label=lst:tonnf,
  caption=
    Converting a complex sentence to NNF,
  language=swift,
  linerange=131-176
]{../../src/ComplexSentence.swift}

\lstinputlisting[
  label=lst:tocnf,
  caption=
    Converting a complex sentence to CNF,
  language=swift,
  linerange=178-228
]{../../src/ComplexSentence.swift}
